\documentclass[journal,12pt,twocolumn]{IEEEtran}
\usepackage{setspace}
\usepackage{gensymb}
\usepackage{caption}
%\usepackage{multirow}
%\usepackage{multicolumn}
%\usepackage{subcaption}
%\doublespacing
\singlespacing
\usepackage{csvsimple}
\usepackage{amsmath}
\usepackage{multicol}
%\usepackage{enumerate}
\usepackage{amssymb}
%\usepackage{graphicx}
\usepackage{newfloat}
%\usepackage{syntax}
\usepackage{listings}
\usepackage{iithtlc}
\usepackage{color}
\usepackage{tikz}
\usetikzlibrary{shapes,arrows}



%\usepackage{graphicx}
%\usepackage{amssymb}
%\usepackage{relsize}
%\usepackage[cmex10]{amsmath}
%\usepackage{mathtools}
%\usepackage{amsthm}
%\interdisplaylinepenalty=2500
%\savesymbol{iint}
%\usepackage{txfonts}
%\restoresymbol{TXF}{iint}
%\usepackage{wasysym}
\usepackage{amsthm}
\usepackage{mathrsfs}
\usepackage{txfonts}
\usepackage{stfloats}
\usepackage{cite}
\usepackage{cases}
\usepackage{mathtools}
\usepackage{caption}
\usepackage{enumerate}	
\usepackage{enumitem}
\usepackage{amsmath}
%\usepackage{xtab}
\usepackage{longtable}
\usepackage{multirow}
%\usepackage{algorithm}
%\usepackage{algpseudocode}
\usepackage{enumitem}
\usepackage{mathtools}
\usepackage{hyperref}
%\usepackage[framemethod=tikz]{mdframed}
\usepackage{listings}
    %\usepackage[latin1]{inputenc}                                 %%
    \usepackage{color}                                            %%
    \usepackage{array}                                            %%
    \usepackage{longtable}                                        %%
    \usepackage{calc}                                             %%
    \usepackage{multirow}                                         %%
    \usepackage{hhline}                                           %%
    \usepackage{ifthen}                                           %%
  %optionally (for landscape tables embedded in another document): %%
    \usepackage{lscape}     


\usepackage{url}
\def\UrlBreaks{\do\/\do-}


%\usepackage{stmaryrd}


%\usepackage{wasysym}
%\newcounter{MYtempeqncnt}
\DeclareMathOperator*{\Res}{Res}
%\renewcommand{\baselinestretch}{2}
\renewcommand\thesection{\arabic{section}}
\renewcommand\thesubsection{\thesection.\arabic{subsection}}
\renewcommand\thesubsubsection{\thesubsection.\arabic{subsubsection}}

\renewcommand\thesectiondis{\arabic{section}}
\renewcommand\thesubsectiondis{\thesectiondis.\arabic{subsection}}
\renewcommand\thesubsubsectiondis{\thesubsectiondis.\arabic{subsubsection}}

% correct bad hyphenation here
\hyphenation{op-tical net-works semi-conduc-tor}

%\lstset{
%language=C,
%frame=single, 
%breaklines=true
%}

%\lstset{
	%%basicstyle=\small\ttfamily\bfseries,
	%%numberstyle=\small\ttfamily,
	%language=Octave,
	%backgroundcolor=\color{white},
	%%frame=single,
	%%keywordstyle=\bfseries,
	%%breaklines=true,
	%%showstringspaces=false,
	%%xleftmargin=-10mm,
	%%aboveskip=-1mm,
	%%belowskip=0mm
%}

%\surroundwithmdframed[width=\columnwidth]{lstlisting}
\def\inputGnumericTable{}                                 %%
\lstset{
%language=C,
frame=single, 
breaklines=true,
columns=fullflexible
}
 

\begin{document}
%
\tikzstyle{block} = [rectangle, draw,
    text width=3em, text centered, minimum height=3em]
\tikzstyle{sum} = [draw, circle, node distance=3cm]
\tikzstyle{input} = [coordinate]
\tikzstyle{output} = [coordinate]
\tikzstyle{pinstyle} = [pin edge={to-,thin,black}]

\theoremstyle{definition}
\newtheorem{theorem}{Theorem}[section]
\newtheorem{problem}{Problem}
\newtheorem{proposition}{Proposition}[section]
\newtheorem{lemma}{Lemma}[section]
\newtheorem{corollary}[theorem]{Corollary}
\newtheorem{example}{Example}[section]
\newtheorem{definition}{Definition}[section]
%\newtheorem{algorithm}{Algorithm}[section]
%\newtheorem{cor}{Corollary}
\newcommand{\BEQA}{\begin{eqnarray}}
\newcommand{\EEQA}{\end{eqnarray}}
\newcommand{\define}{\stackrel{\triangle}{=}}

\bibliographystyle{IEEEtran}
%\bibliographystyle{ieeetr}

\providecommand{\nCr}[2]{\,^{#1}C_{#2}} % nCr
\providecommand{\nPr}[2]{\,^{#1}P_{#2}} % nPr
\providecommand{\mbf}{\mathbf}
\providecommand{\pr}[1]{\ensuremath{\Pr\left(#1\right)}}
\providecommand{\qfunc}[1]{\ensuremath{Q\left(#1\right)}}
\providecommand{\sbrak}[1]{\ensuremath{{}\left[#1\right]}}
\providecommand{\lsbrak}[1]{\ensuremath{{}\left[#1\right.}}
\providecommand{\rsbrak}[1]{\ensuremath{{}\left.#1\right]}}
\providecommand{\brak}[1]{\ensuremath{\left(#1\right)}}
\providecommand{\lbrak}[1]{\ensuremath{\left(#1\right.}}
\providecommand{\rbrak}[1]{\ensuremath{\left.#1\right)}}
\providecommand{\cbrak}[1]{\ensuremath{\left\{#1\right\}}}
\providecommand{\lcbrak}[1]{\ensuremath{\left\{#1\right.}}
\providecommand{\rcbrak}[1]{\ensuremath{\left.#1\right\}}}
\theoremstyle{remark}
\newtheorem{rem}{Remark}
\newcommand{\sgn}{\mathop{\mathrm{sgn}}}
\providecommand{\abs}[1]{\left\vert#1\right\vert}
\providecommand{\res}[1]{\Res\displaylimits_{#1}} 
\providecommand{\norm}[1]{\lVert#1\rVert}
\providecommand{\mtx}[1]{\mathbf{#1}}
\providecommand{\mean}[1]{E\left[ #1 \right]}
\providecommand{\fourier}{\overset{\mathcal{F}}{ \rightleftharpoons}}
%\providecommand{\hilbert}{\overset{\mathcal{H}}{ \rightleftharpoons}}
\providecommand{\system}{\overset{\mathcal{H}}{ \longleftrightarrow}}
	%\newcommand{\solution}[2]{\textbf{Solution:}{#1}}
\newcommand{\solution}{\noindent \textbf{Solution: }}
\newcommand{\myvec}[1]{\ensuremath{\begin{pmatrix}#1\end{pmatrix}}}
\providecommand{\dec}[2]{\ensuremath{\overset{#1}{\underset{#2}{\gtrless}}}}
\DeclarePairedDelimiter{\ceil}{\lceil}{\rceil}
%\numberwithin{equation}{subsection}
\numberwithin{equation}{section}
%\numberwithin{problem}{subsection}
%\numberwithin{definition}{subsection}
\makeatletter
\@addtoreset{figure}{section}
\makeatother

\let\StandardTheFigure\thefigure
%\renewcommand{\thefigure}{\theproblem.\arabic{figure}}
\renewcommand{\thefigure}{\thesection}


%\numberwithin{figure}{subsection}

%\numberwithin{equation}{subsection}
%\numberwithin{equation}{section}
%\numberwithin{equation}{problem}
%\numberwithin{problem}{subsection}
\numberwithin{problem}{section}
%%\numberwithin{definition}{subsection}
%\makeatletter
%\@addtoreset{figure}{problem}
%\makeatother
\makeatletter
\@addtoreset{table}{section}
\makeatother

\let\StandardTheFigure\thefigure
\let\StandardTheTable\thetable
\let\vec\mathbf
%%\renewcommand{\thefigure}{\theproblem.\arabic{figure}}
%\renewcommand{\thefigure}{\theproblem}

%%\numberwithin{figure}{section}

%%\numberwithin{figure}{subsection}



\def\putbox#1#2#3{\makebox[0in][l]{\makebox[#1][l]{}\raisebox{\baselineskip}[0in][0in]{\raisebox{#2}[0in][0in]{#3}}}}
     \def\rightbox#1{\makebox[0in][r]{#1}}
     \def\centbox#1{\makebox[0in]{#1}}
     \def\topbox#1{\raisebox{-\baselineskip}[0in][0in]{#1}}
     \def\midbox#1{\raisebox{-0.5\baselineskip}[0in][0in]{#1}}

\vspace{3cm}

\title{ 
	\logo{
Statistical Decision Theory
	}
}

\author{ G V V Sharma$^{*}$% <-this % stops a space
	\thanks{*The author is with the Department
		of Electrical Engineering, Indian Institute of Technology, Hyderabad
		502285 India e-mail:  gadepall@iith.ac.in. All content in this manual is released under GNU GPL.  Free and open source.}
	
}	

\maketitle

\tableofcontents

\bigskip

\renewcommand{\thefigure}{\theenumi}
\renewcommand{\thetable}{\theenumi}


\begin{abstract}
	
This manual provides an introduction to statistical decision theory.
\end{abstract}
%
\section{Mean Square Error}
\begin{enumerate}[label=\thesection.\arabic*
,ref=\thesection.\theenumi]
\item Let $\brak{\vec{X},Y}$ be an input/output dataset, whose relation is unknown.  If 
\begin{align}
\hat{Y} = f \brak{\vec{X}}
\end{align}
be the estimate of $Y$, the {\em loss} in the estimate is defined as
\begin{align}
L\brak{Y,\hat{Y}} = L\brak{f}
\end{align}

\item Let
\begin{align}
L\brak{f} = \brak{Y - \hat{Y}}^2
\end{align}
%
Then
\begin{align}
E\sbrak{L\brak{f}} 
\end{align}
%
is defined as the {\em mean square error} (MSE).
Show that
\begin{align}
E\sbrak{L(f)} = E_\vec{X}\cbrak{E_Y\sbrak{Y - f\brak{\vec{x}}}^2|\vec{X}= \vec{x}}
\label{eq:mse_cond}
\end{align}
%
\item Let 
\begin{align}
c= f\brak{\vec{x}}
\end{align}
%
Using \eqref{eq:mse_cond}
\begin{multline}
\min E\sbrak{L(f}) = \min L(f)|\vec{X} 
\\
= \min_{c}E_Y\cbrak{\sbrak{Y - c}^2|\vec{X}= \vec{x}}
\end{multline}
%
Show that 
\begin{multline}
L(f)|\vec{X} = E_Y\cbrak{\sbrak{Y - c}^2|\vec{X}= \vec{x}} 
\\
= -2cE_Y\cbrak{Y|\vec{X}= \vec{x}}
+E_Y\cbrak{Y^2 
|\vec{X}= \vec{x}}+ c^2
\end{multline}
\item $L(f)$ is minimum when 
\begin{align}
\frac{d}{dc}L(f)|\vec{X}= 0.
\end{align}
%
Show that this results in 
\begin{align}
c = f\brak{\vec{x}}= E\sbrak{Y|X=\vec{x}}
\label{eq:opt_mse}
\end{align}
$f$ is known as the {\em regression} function.
\end{enumerate}
\section{Bayes Classifier}
\begin{enumerate}[label=\thesection.\arabic*
,ref=\thesection.\theenumi]
\item Let $\brak{\vec{X},\vec{G}}$ be an input/output dataset, whose relation $f$ is unknown. Also
\begin{align}
\vec{g}\in\vec{G} = \cbrak{\vec{g}_k}_{k=1}^{K}
%\\
%\hat{\vec{G}} &= f\brak{\vec{X}}
\end{align}
Let 
\begin{align}
C\brak{\vec{g}_k,\vec{g}_l} =
\begin{cases}
1 & k= l
\\
0 & k\ne l
\end{cases}
\end{align}
%
where $\vec{g}_i$ are different classes of output data. Thus $C$ is a {\em correctness} metric.
\item Show that 
\begin{align}
\max_{\vec{g}\in\vec{G}}E\sbrak{C\brak{\vec{G},f\brak{\vec{X}}}} 
= \max_{\vec{g}\in\vec{G}}p\brak{\vec{g}|\vec{X}=\vec{x}} 
\end{align}
\solution In the above, 
\begin{multline}
\max_{\vec{g}\in\vec{G}}E\sbrak{C\brak{\vec{G},f\brak{\vec{X}}}} 
\\
= \max_{\vec{g}\in\vec{G}}E_\vec{X}\sbrak{E_{\vec{G}}\cbrak{C\brak{\vec{G},f\brak{\vec{x}}}}} 
\end{multline}

\begin{align}
&=\max_{\vec{g}\in\vec{G}}\sum_{k=1}^{K}L\cbrak{\vec{g}_k,\vec{g}}p\brak{\vec{g}_k|\vec{X}=\vec{x}} 
\end{align}
where
\begin{align}
\sum_{k=1}^{K}p\brak{\vec{g}_k|\vec{X}=\vec{x}} = 1
\end{align}
\item Repeat the exercise for the least squares method.
\end{enumerate}
%
\section{Exercises}
\begin{enumerate}[label=\thesection.\arabic*
,ref=\thesection.\theenumi]
\item Explain how \eqref{eq:opt_mse} can be used to obtain the  Nearest 
Neighbour approximation.
\item Repeat the exercise for the least squares method.
\end{enumerate}

\end{document}
